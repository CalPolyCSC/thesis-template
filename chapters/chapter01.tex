\chapter{Example Usage}
    This chapter is an introduction to the various elements of a thesis that might need to be created.
    Any additional packages beyond what is already included in the main \LaTeX\ file should be put into the \textbf{preamble.tex} file located in the \textbf{frontmatter} project directory.

\section{Math}
    Displaying math relations is one of the areas that \LaTeX\ shines.
    While this is not a complete presentation of the capabilities, there are a few uses that are worth presenting as a starting point.

\subsection{General Equations}
    This section shows how to create a variety of equations and cite them.
    \Cref{eq:quadratic_formula} is an example of a simple equation.
    \begin{equation} \label{eq:quadratic_formula}
        x=\frac{-b\pm\sqrt{b^2-4ac}}{2a}
    \end{equation}
    For equations that are not to be numbered, then can use the * version of the math commands, such as
    \begin{equation*}
        y=mx+b
    \end{equation*}
    Note that since this equation is not numbered, it cannot be referenced in the text because it has no label.

    If you have multiple equations that are related then can use the following
    \begin{subequations}
        \begin{align}
            \begin{split}
                q\left(x\right)&\coloneq\frac{6x^2}{2x} \\
                &=3x
            \end{split} \\
            q\left(1\right)
                &=3\cdot 1=3 \\
            q\left(0\right)
                &=\lim_{x\to 0} q=0
        \end{align}
    \end{subequations}
    In the case that some of the equations should not be numbered, then can use \textbf{notag} on the equations that should not be numbered
    \begin{subequations}
        \begin{align}
            \begin{split}
                q\left(x\right)
                    &\coloneq\frac{6x^2}{2x} \\
                    &=3x
            \end{split} \\
            q\left(1\right)
                &=3\cdot 1=3 \notag \\
            q\left(0\right)
                &=\lim_{x\to 0} q=0
        \end{align}
    \end{subequations}

    For one equation that splits over multiple lines, \cref{eq:multi_line} can be used as a reference
    \begin{equation} \label{eq:multi_line}
    \begin{split}
        x&=\frac{-b\pm\sqrt{b^2-4ac}}{2a}
          =\frac{-b\pm\sqrt{b^2-4ac}}{2a}\left(\frac{-b\mp\sqrt{b^2-4ac}}{-b\mp\sqrt{b^2-4ac}}\right) \\
         &=\frac{b^2-\left(b^2-4ac\right)}{2a\left(-b\mp\sqrt{b^2-4ac}\right)}
          =\frac{4ac}{2a\left(-b\mp\sqrt{b^2-4ac}\right)} \\
         &=-\frac{2c}{b \pm \sqrt{b^2-4ac}}
        \end{split}
    \end{equation}

\subsection{Functions and Their Names}
    It is important to remember that math functions, such as sine and cosine, have their own definitions, and are typeset differently compared to variables.
    \begin{equation}
        \sin\left(2\alpha\right)=2\sin\left(\alpha\right)\cos\left(\alpha\right)
    \end{equation}
    Any math function that is represented by multiple characters should follow the same formatting.
    To create custom math functions, use the \textbf{DeclareMathOperator} command.

\subsection{Representing Vectors}
    Vectors are represented a variety of ways in different fields of study.
    A few of the most common methods are presented here.

\subsubsection{Vectors as Bold Symbols}
    A three-dimensional vector has the form
    \begin{equation}
        \bm{x}
            =\begin{bmatrix}
                x_0 \\ x_1 \\ x_2
             \end{bmatrix}
    \end{equation}
    while a row vector looks like
    \begin{equation}
        \bm{\alpha}
            =\begin{bmatrix}
                \alpha_0 & \alpha_1 & \alpha_2
             \end{bmatrix}
    \end{equation}
    The same process can be used to represent a matrix
    \begin{equation}
        \bm{\mu}
            =\begin{bmatrix}
                \bm{\mu}_0 & \bm{\mu}_1 & \bm{\mu}_{2}
             \end{bmatrix}
            =\begin{bmatrix}
                \mu_{0,0} & \mu_{0,1} & \mu_{0,2} \\
                \mu_{1,0} & \mu_{1,1} & \mu_{1,2} \\
                \mu_{2,0} & \mu_{2,1} & \mu_{2,2} \\
             \end{bmatrix}
    \end{equation}

\subsubsection{Vectors with Arrows Above}
    A three-dimensional vector has the form
    \begin{equation}
        \vec{x}
            =\begin{bmatrix}
                x_0 \\ x_1 \\ x_2
             \end{bmatrix}
    \end{equation}
    while a row vector looks like
    \begin{equation}
        \vec{\alpha}
            =\begin{bmatrix}
                \alpha_0 & \alpha_1 & \alpha_2
             \end{bmatrix}
    \end{equation}
    The same process can be used to represent a matrix
    \begin{equation}
        \vec{\mu}_i
            =\begin{bmatrix}
                \vec{\mu}_0 & \vec{\mu}_1 & \vec{\mu}_{2}
             \end{bmatrix}
            =\begin{bmatrix}
                \mu_{0,0} & \mu_{0,1} & \mu_{0,2} \\
                \mu_{1,0} & \mu_{1,1} & \mu_{1,2} \\
                \mu_{2,0} & \mu_{2,1} & \mu_{2,2} \\
             \end{bmatrix}
    \end{equation}

\subsubsection{Vectors with Lines Below}
    A three-dimensional vector has the form
    \begin{equation}
        \underline{x}
            =\begin{bmatrix}
                x_0 \\ x_1 \\ x_2
             \end{bmatrix}
    \end{equation}
    while a row vector looks like
    \begin{equation}
        \underline{\alpha}
            =\begin{bmatrix}
                \alpha_0 & \alpha_1 & \alpha_2
             \end{bmatrix}
    \end{equation}
    The same process can be used to represent a matrix
    \begin{equation}
        \underline{\underline{\mu}}
            =\begin{bmatrix}
                \underline{\mu_0} & \underline{\mu_1} & \underline{\mu_2}
             \end{bmatrix}
            =\begin{bmatrix}
                \mu_{0,0} & \mu_{0,1} & \mu_{0,2} \\
                \mu_{1,0} & \mu_{1,1} & \mu_{1,2} \\
                \mu_{2,0} & \mu_{2,1} & \mu_{2,2} \\
             \end{bmatrix}
    \end{equation}
